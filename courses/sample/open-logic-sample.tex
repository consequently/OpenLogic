% Open Logic Project
%
% driver file open-logic-sample.tex to produce text on letter-size paper
% with standard layout and margins

% We use the memoir class for maximal flexibility of layout, but any
% class will do

\documentclass[a4paper]{memoir}

% \olpath has to point to the location of the OLP main
% directory/folder.  We're compiling from subdirectory courses/sample,
% so the main directory is two levels up.
\newcommand{\olpath}{../../}

% load all the Open Logic definitions. This will also load the
% local definitions in open-logic-sample-config.sty
\input{\olpath/sty/open-logic.sty}

% we want all the problems deferred to the end
\input{\olpath/sty/open-logic-defer.sty}

% let's set the whole thing in Palatino, with Helvetica for
% sans-serif, and spread the lines a bit to make the text more
% readable

%\usepackage{mathpazo}
\usepackage{Alegreya,AlegreyaSans}
%\usepackage[scaled=0.95]{helvet}
\linespread{1.05}

\begin{document}

% First we make a titlepage

\begin{titlingpage}
\begin{raggedleft}
\fontsize{52pt}{2em}\selectfont\bfseries\sffamily
Sample\\[.5ex] 
Logic\\[.5ex] 
Text
\vskip 4ex
\normalfont\Huge\textbf{\href{http://openlogicproject.org/}{Open Logic Project}}

\end{raggedleft}

\vfill

% oluselicense generates a license mark that a) licenses the result
% under a CC-BY licence and b) acknowledges the original source (the
% OLP).  Acknowledgment of the source is a requirement under the
% conditions of the CC-BY license used by the OLP, but you are not
% required to license the product itself under CC-BY.

\oluselicense
% Title of this version of the OLT with link to source
{\href{https://github.com/OpenLogicProject/OpenLogic/tree/master/courses/sample}{\textit{Sample Logic Text}}}
% Author of this version
{\href{http://openlogicproject.org/}{OLP}}
\end{titlingpage}

\frontmatter
\pagestyle{ruled}

\tableofcontents*

\mainmatter

% olimport includes an entire part

\olimport*[sets-functions-relations]{sets-functions-relations}

% you can also import individual chapters, but then don't forget to
% include part headings

\part{First-order Logic}

\olimport*[first-order-logic/introduction]{introduction}

\olimport*[first-order-logic/syntax-and-semantics]{syntax}

\olimport*[first-order-logic/syntax-and-semantics]{semantics}

\olimport*[first-order-logic/models-theories]{models-theories}

% For a proof system, we'll do only natural deduction. But some of the
% texts will refer to other proof systems unless you set some tags in
% the config.sty file.  So make sure those are set. We'll include
% these sections by hand so we can add a couple of sections from the
% proof-systems chapter. Of course, if new sections are added or
% sections are moved or renamed in the main repository, this may
% break.

\chapter{Natural Deduction}

\olimport*[first-order-logic/proof-systems]{introduction}

\olimport*[first-order-logic/proof-systems]{natural-deduction}

\olimport*[first-order-logic/natural-deduction]{rules-and-proofs}

\olimport*[first-order-logic/natural-deduction]{propositional-rules}

% We'll reorder things: let's do propositional examples first and then
% go back to quantifiers

\olimport*[first-order-logic/natural-deduction]{derivations}

\olimport*[first-order-logic/natural-deduction]{proving-things}

\olimport*[first-order-logic/natural-deduction]{quantifier-rules}

\olimport*[first-order-logic/natural-deduction]{proving-things-quant}

\olimport*[first-order-logic/natural-deduction]{proof-theoretic-notions}

\olimport*[first-order-logic/natural-deduction]{provability-consistency}

\olimport*[first-order-logic/natural-deduction]{provability-propositional}

\olimport*[first-order-logic/natural-deduction]{provability-quantifiers}

\olimport*[first-order-logic/natural-deduction]{soundness}

\olimport*[first-order-logic/natural-deduction]{identity}

\olimport*[first-order-logic/natural-deduction]{soundness-identity}

% Chapters should end with \OLEndChapterHook. They will automatically
% if you include entire parts or chapters. Here we did a chapter ``by
% hand'' so we should add \OLEndChapterHook by hand too

\OLEndChapterHook

\olimport*[first-order-logic/completeness]{completeness}

\olimport*[first-order-logic/beyond]{beyond}

% OLEndPartHook should come at the end of each
% part.

\OLEndPartHook

% Include some more chapters and parts

\olimport*[turing-machines]{turing-machines}

% Part: incompleteness

\part{Computability and Incompleteness}

% include intro to recursive function from computability part

\olimport*[computability/recursive-functions]{recursive-functions}

\olimport*[incompleteness/arithmetization-syntax]{arithmetization-syntax}

\olimport*[incompleteness/representability-in-q]{representability-in-q}

% leave out this part -- it depends on computability theory chapter
% \olimport[incompleteness/theories-computability]{theories-computability}

\olimport*[incompleteness/incompleteness-provability]{incompleteness-provability}

\OLEndPartHook

\stopproblems

% Ok, that's it. Now for the appendices

\appendix

\olimport*[methods]{methods}

\olimport*[history/biographies]{biographies}

% now typeset all the problems as an appendix. If you want problems at
% the end of each chapter, delete this part and put
% \problemsperchapter in the preamble

\chapter{Problems}

\printproblems

\backmatter

% If you include any chapters from the history part, you have to print
% the Photo Credits. 

\photocredits

% Include the bibliography

\bibliographystyle{\olpath/bib/natbib-oup}
\bibliography{\olpath/bib/open-logic}

\end{document}

